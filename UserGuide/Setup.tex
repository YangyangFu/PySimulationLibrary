\section{Setting up PySimLib}

The following will show you how to set up PySimLib, while mainly focusing on Windows platforms.
Installation instructions for systems based on Ubuntu are also given.
Due to the variety of Linux systems we can't provide installation instructions for all of them.
Unfortunately we currently have no access to a Mac computer and can therefore not provide installation instructions, nor do we know if PySimLib works on Mac at all.
We would be pleased to hear feedback on that.

\subsection{Note on versions}

PySimLib was developed for Python 3, in particular version 3.5.
The library might also work with previous versions of Python 3 but surely not with Python 2.
By us, PySimLib has only been tested in Python 3.5.

PySimLib currently supports the simulators listed in table \ref{simulatorsList}.
It should be noted that these simulators exist in different versions, which differentiate in terms of features etc.
We have listed the constellations of operating system and simulator version under which we tested PySimLib in table \ref{testTable}.

\begin{table}[h]
	\centering
	\begin{tabular}{| l |}
		\hline
		Dymola \\ \hline
		OpenModelica \\ \hline
		MATLAB/Simulink \\ \hline
	\end{tabular}
	\caption{List of simulators supported by PySimLib}
	\label{simulatorsList}
\end{table}

\begin{table}[h]
	\centering
	\begin{tabular}{c | c | c | c | c |}
		%header
		&
		\rotatebox{90}{Windows 7 64 bit} &
		\rotatebox{90}{Windows 10 64 bit} &
		\rotatebox{90}{Ubuntu 14.10} &
		\rotatebox{90}{Ubuntu 16.04}
		\\ \hline
		
		%							Win7 x64   & Win10 x64  & Ubuntu 14.10	& Ubuntu 16.04
		Dymola 2013 			&	\checkmark &			& 				& 				\\ \hline
		OpenModelica 1.9.6 		&	           & 			& \checkmark	& \checkmark 	\\ \hline
		MATLAB/Simulink R2013a 	&			   & 			& 				& \checkmark 	\\ \hline
		
	\end{tabular}
	\caption{Testing Constellations of Operating Systems and Simulators using PySimLib}
	\label{testTable}
\end{table}




\subsection{Command line}

\TODO{wie man ne command line öffnet}





\subsection{Installing Python}

\subsubsection{Windows}

\TODO{bild mit python was angeklickt sein muss}

\subsubsection{Ubuntu}
Python should already be installed on Ubuntu.
If not, install it by running the following commands:
\lstset{language=bash}
\begin{lstlisting}
sudo apt-get install python3
sudo apt-get install python3-pip
\end{lstlisting}
If you are unsure whether Python is installed or not, run the commands anyways.
In case Python is already installed, apt-get is going to inform you and nothing is going to be done.





\subsection{Installing a Simulator}

In order to simulate models using the PySimLib you will need a simulator that can simulate your models.
Note that PySimLib is not a simulator itself but can communicate with several ones and provides a common interface to do so.
Please follow the manual of the simulator of your choice in order to set it up correctly.
Be sure that it is working properly before continuing with PySimLib.
Again, the supported simulators are mentioned in Table \ref{supportedSimulators}.

You can install (or remove) additional simulators any time but whenever you do that, you have to reconfigure PySimLib.
See subsection \ref{configuringPySimLib} for configuring PySimLib.





\subsection{Installing PySimLib}

PySimLib can be installed very simply using pip.
In order to determine the correct pip version try entering the following commands:
\lstset{language=bash}
\begin{lstlisting}
pip3.5
pip3
pip
\end{lstlisting}

Remember the first one that worked and take it for all following commands where "pipxyz" is used.
\\
To install PySimLib execute the following commands (\textbf{you might need advanced privileges}):
\begin{lstlisting}
pipxyz install zmq
pipxyz install PySimLib
\end{lstlisting}






\subsection{Configuring PySimLib}
\label{configuringPySimLib}

\TODO{schreiben}